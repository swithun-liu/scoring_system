% !TEX TS-program = XeLaTeX
% !TEX encoding = UTF-8 Unicode

\chapter{需求分析}
\label{chap07}
\defaultfont
~\\
\section{可行性分析}
~\\
\subsection{技术可行性分析}
\subsection{经济可行性分析}
\subsection{操作可行性分析}
~\\
\section{业务流程分析}

 (1) 管理员通过教职员学生管理系统与论文评审评分系统联动导入学生,教师信息。

(2) 学生上传论文

学生上传论文,待评审导师进行检查。

(3) 检查论文

教师检查论文,若论文未达到审批要求,教师可进行留言并贴上缺陷标签,待学生修改论文并重新上传论文。

(4) 论文打分

若论文达到审批要求,教师可对论文进行打分。

(5) 统计得分情况

管理员可查看所有学生论文得分比例。

(5) 统计教师评审量,评审情况

管理员可查看各个教师总评审量,已完成评审比例,供后续任务量分配做参考。

(6) 统计论文中出现的问题

管理可查看所有缺陷标签以及标签引用量。
\begin{figure}[h]
	\centering
	\includegraphics[scale = 0.6]{out/uml/流程图/系统总体活动流程图/系统总体活动流程图.png}
	\caption{\song\wuhao 系统总体活动流程图}
\end{figure}


\section{系统功能需求分析}

\subsection{系统总体功能}

如图\ref{system-usecase}所示,系统的主要功能有用户登陆功能(学生,教师,管理员),论文上传功能(学生),论文评审评分功能(教师),统计评审情况功能(管理员)。

\begin{figure}[h]
	\centering
	\includegraphics[scale = 0.6]{out/uml/用例图/系统总体功能用例图/系统总体功能用例图.png}
	\caption{\song\wuhao 系统总体功能用例图}
	\label{system-usecase}
\end{figure}

\subsection{用户登录}

如图\ref{login-usecase}所示,用户填入用户名,用户密码,登陆人员类型(学生,教师,管理员),点击提交,后台从数据库验证用户名,用户密码是否正确,如果正确,则为用户授权,不同用户拥有不同的权力,将包含用户信息的Token返回给前端,前端将Token保存在浏览器,之后的通信都需要携带Token,后台可以从Token中获取用户信息用来验证用户合法性。

\begin{figure}[h]
	\centering
	\includegraphics[scale = 0.45]{out/uml/用例图/1-用户登录用例图/1-用户登录用例图.png}
	\caption{\song\wuhao 用户登录用例图}
	\label{login-usecase}
\end{figure}

\subsection{论文上传}

如图\ref{upload-usecase}所示,管理员可以选择是否开放论文上传通道,在论文开放通道期间,学生可以使用用户名,用户密码登陆系统,选择自己的论文进行上传。且当自己的论文没有通过教师的评审时,根据老师的评论和问题标签,学生修改自己的论文并更新自己的文档。

\begin{figure}[h]
	\centering
	\includegraphics[scale = 0.6]{out/uml/用例图/2-论文上传/2-论文上传.png}
	\caption{\song\wuhao 论文上传用例图}
	\label{upload-usecase}
\end{figure}

\subsection{论文下载}

如图\ref{download-usecase}所示,学生,教师,管理员分别可以从各自的页面下载论文的最新版本进行查看。

\begin{figure}[h]
	\centering
	\includegraphics[scale = 0.6]{out/uml/用例图/5-论文下载/5-论文下载.png}
	\caption{\song\wuhao 论文下载用例图}
	\label{download-usecase}
\end{figure}

\subsection{论文评审评分}

如图\ref{scoring-usecase}所示,教师可以从学生管理页面浏览论文列表,下载之后教师进行阅读和评审,如果评审未通过,教师可以选择论文进行评论,讲解论文出现的问题,同时可以为论文添加问题标签,便于学生快速了解自己的问题所在并进行论文的修改,学生在修改论文之后,重新上传自己的论文,教师重新阅读,达到评审标准之后,教师可以为论文评分,学生可以登陆系统查看论文得分。

\begin{figure}[h]
	\centering
	\includegraphics[scale = 0.6]{out/uml/用例图/3-论文评审评分/3-论文评审评分.png}
	\caption{\song\wuhao 论文评审评分用例图}
	\label{scoring-usecase}
\end{figure}

\subsection{统计评审情况}

如图\ref{statistic=usecase}所示,管理员可以登陆系统查看所有论文的得分统计(分类有95以上,90-95分,80-90分,70-80分,60-70分,60分以下,待评分),管理员可以通过该比例把控论文评审进程。管理员还可以查看教师的评审工作统计(分类有总分配学生数量,总分类论文数量,已评分论文数量,尚未评审论文数量),管理员可以根据教师的任务完成情况进行之后的任务分配。

\begin{figure}[h]
	\centering
	\includegraphics[scale = 0.6]{out/uml/用例图/4-统计评审情况/4-统计评审情况.png}
	\caption{\song\wuhao 统计评审情况用例图}
	\label{statistic=usecase}
\end{figure}
