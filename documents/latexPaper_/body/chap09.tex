% !TEX TS-program = XeLaTeX
% !TEX encoding = UTF-8 Unicode

\chapter{实现与测试}
\label{chap07}
\defaultfont

\section{系统环境}

\subsection{硬件环境}

\begin{enumerate}
    \item CPU:Intel(R) Core(TM) i7-7700HQ CPU @ 2.80GHz   2.80 GHz
    \item 内存:16.0 GB
    \item 硬盘容量:500GB
\end{enumerate}

\subsection{软件环境}

\begin{enumerate}
    \item 操作系统:Windows 10 家庭中文版(20H2)作为编辑环境+ WSL2(Ubuntu 20.04)作为开发环境
    \item 数据库:mysql Ver 8.0.23-0ubuntu0.20.04.1 for Linux on x86\_64 (Ubuntu)
    \item 客户端:主流浏览器
\end{enumerate}


\section{持久化}
\subsection{Spring Date \& JPA Hibernate}

\begin{enumerate}
    \item 使用“逻辑模型工具”(例如Navicat Data Module 3)建立模型。
    \item 将模型同步至数据库
    \item 使用IDEA持久化工具生成实体类
    \item 根据逻辑模型为实体类添加注解(实现级联关系)
\end{enumerate}

关键问题如下:

\begin{enumerate}
    \item 多对一,一对多关系实现\\
          这种关系一共有四种实现方式
          \begin{enumerate}
              \item 单向\lstinline[language = Java]| @OneToMany |绑定
              \item 带有\lstinline[language = Java]| @JoinColumn |的单向\lstinline[language = Java]| @OneToMany |绑定
              \item 双向\lstinline[language = Java]| @OneToMany , @ManyToOne|绑定
              \item 不带有\lstinline[language = Java]| ManyToOne |的双向绑定
          \end{enumerate}
          本系统采用第三种方式
          \begin{lstlisting} [language = Java]
    public class A { //  一方
        ...
        @OneToMany(...)
        private List<B> bs;
        ...
    }
    public class B { // 多方
        @ManyToOne
        @JoinColumn(name = "a_id", referencedColumnName = "id")
        private A a;
    }
\end{lstlisting}
    \item 级联更新
          \begin{enumerate}
              \item 在“一方”(A)中 \lstinline[language = Java]| @OneToMany(cascade = CascadeType.ALL) |
              \item "深拷贝"A中的 \lstinline[language = Java]| BCopy = List<B> |
              \item 清空A中的 \lstinline[language = Java]| List<B> |
              \item \lstinline[language = Java]| BRepository.deleteAll(BCopy) |
              \item \lstinline[language = Java]| ARepository.save(A) |
          \end{enumerate}
    \item 级联查询
          \begin{enumerate}
              \item Dao层:继承JpaSpecificationExecutor
                    \begin{lstlisting} [language = Java]
    public interface ARepository extends 
        JpaRepository<A, Integer>, 
        JpaSpecificationExecutor<AEntity> // 注意这里
\end{lstlisting}
              \item Service层使用:
                    \begin{lstlisting} [language = Java]
    aRepository.findAll(new Specification<A>() {
        @Override
        // Hibernate根据此方法生成过滤语句
        public Predicate toPredicate(
            Root<UnfinishedPlanEntity> root, 
            CriteriaQuery<?> query, 
            CriteriaBuilder criteriaBuilder) {
          // 实现逻辑
        }
    })             
\end{lstlisting}
                    涉及的重要的类:
                    \begin{enumerate}
                        \item \lstinline[language = Java]| javax.persistence.criteria.Path |:通过实体类属性名获取字段值
                        \item \lstinline[language = Java]| javax.persistence.criteria.Join |:通过实体类属性名级联
                        \item \lstinline[language = Java]| javax.persistence.criteria.CriteriaBuilder |:拼接生成过滤
                    \end{enumerate}
          \end{enumerate}
\end{enumerate}

\section{登陆\&验证授权模块}

\subsection{JWT优势}

随着Web应用规模的逐渐扩大,传统的基于Session和cookie的身份验证技术逐渐显现出它的弊端。
随着服务器的不断增加,由于多个请求可以被分发给不同的服务器,那么此时,服务器该如何那些请求来自同一个用户,对此有两种解决方式
\begin{enumerate}
    \item 多个服务器之间同步用户状态,这样无论用户的请求分发到哪一个服务器,都可以操作用户状态
    \item 判断来自同一个用户的请求,将同一个用户的请求一直分发到同一个服务器
\end{enumerate}
很显然,无论是上述哪种方式,随着用户量的增长开销都会变得很大。
例如第一种方式可以使用会话复制实现,Sesson复制性能也会随着服务器的增加而急剧下降。\cite{.2019h}

JWT(JSON Web Token)的优势在于
\begin{enumerate}
    \item 将用户认证信息保存在客户端,减轻服务器的存储压力。
    \item 提供无状态的身份认证——不需要服务器端多端同步用户状态,利于分布式应用开发。\cite{.2019h}
\end{enumerate}


\subsection{使用JWT\&Spring Security 实现验证和授权}

\begin{figure}[h]
    \centering
    \includegraphics[scale = 0.3, angle = 90]{out/uml/时序图/时序图-authentication/时序图-authentication.png}
    \caption{\song\wuhao 登陆验证授权时序图}
\end{figure}

关键问题如下:
\begin{enumerate}
    \item 多用户表使用Spring Security\\
    验证授权涉及到的关键类:
    \begin{enumerate}
        \item \lstinline[language = Java]| Authentication |:用户认证信息。
        \item \lstinline[language = Java]| UsernamePasswordAuthenticationToken |:\lstinline[language = Java]| Authentication |的实现类,用于登陆验证,最为常用。
        \item \lstinline[language = Java]| AuthenticationProvider |:认证器,不同的\lstinline[language = Java]| AuthenticationProvider | 处理不同的 \lstinline[language = Java]| Authentication |
        \item \lstinline[language = Java]| DaoAuthenticationProvider |:\lstinline[language = Java]| AuthenticationProvider |的实现类,用于处理\lstinline[language = Java]| UsernamePasswordAuthenticationToken |
        \item \lstinline[language = Java]| ProviderManager |:用于管理\lstinline[language = Java]| AuthenticationProvider |
        \item \lstinline[language = Java]| UserDetailService |:为\lstinline[language = Java]| DaoAuthenticationProvider |提供从数据库查询用户的功能,用于做密码对比。
    \end{enumerate}
\end{enumerate}