% !TEX TS-program = XeLaTeX
% !TEX encoding = UTF-8 Unicode

\chapter*{\hfill 修改记录 \hfill}
\addcontentsline{toc}{chapter}{修改记录}
\defaultfont
\linespread{1.25}
修改是论文写作过程中不可或缺的重要步骤,是提高论文质量的有效环节。修改的过程其实就是“去伪存真”、去糟粕取精华使论文不断“升华”的过程。

以下内容要求放到毕业设计(论文)修改记录中:

(1) 毕业设计(论文)题目修改

% {\textbf {第一次修改记录:}}:(没有可删除,后面记录依次递进)

% 原题目:

% 修稿后题目:

(2) 指导教师变更

% {\textbf {第二次修改记录:}}:(没有可删除,后面记录依次递进)

% 原指导教师:******更改后指导教师:******

(3) 校外毕业设计(论文)时间节点记录

% {\textbf {第三次修改记录:}}:(没有可删除,后面记录依次递进)

% 本人于2019年1月申请到******大学做毕业设计(论文),指导教师为:******

% 校内指导教师为:******。2019年*月*日回到学校。

(4) 毕业设计(论文)内容重要修改记录

包括:指导教师要求的重大修改,评阅教师要求的修改,答辩委员会提出的修改意见以及检测后的修改记录等。

{\textbf {第一次修改记录:}}

所有表,{\textbf{修改前:}}非三线表

{\textbf{修改后:}}三线表

{\textbf {第二次修改记录:}}

所有较长代码块,{\textbf{修改前:}}

{\textbf{修改后:}}删除

{\textbf {第三次修改记录:}}

所有有序列表,{\textbf{修改前:}}非顶格

{\textbf{修改后:}}顶格且下文为正常段落(首行缩进)

(5) 毕业设计(论文)外文翻译修改记录

(6) 毕业设计(论文)正式检测重复比

\hspace*{18em}记录人(签字):\\
\hspace*{19em}指导教师(签字):