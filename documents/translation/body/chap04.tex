% !TEX TS-program = XeLaTeX
% !TEX encoding = UTF-8 Unicode

\chapter{基于FPGA加速算法}
\label{chap04}
\defaultfont
在这一部分中,我们将展示如何通过FPGA原型板来提高我们的人脸识别算法的时间性能。在实现它之前,首先要探究该算法的数据并行性,在此基础上,便可以设计出硬件加速器。
\begin{figure}
	\centering
	\caption{\song\wuhao 基于不同\u 值,查询恩连第一行数据的直方图和标准差}

\end{figure}
可以通过两种方式来充分利用该算法所具备的较高数据并行性:第一,查询面的所有行都是独立的。这允许每一行与其他行并行处理。其次,直方图、标准差和距离指数的计算过程是独立的。因此,可以在不同的阶段执行。

基于Xilinx1的原型化FPGA-Zed板是实现该算法的良好选择之一。该板包含ZYNQ-7000全可编程SoC-FPGA[11],具有13300个逻辑片、220个DSP48E1s和140个BlockRAMs。Zynq芯片结合了ARM双核Cortex–A9MPCore处理系统(PS)和Xilinx28nm Programmable Logic(PL),用户可以在同一芯片上创建自己的IP核。这两个部分通过工业标准高级可扩展接口(AXI)互连进行通信,在设备的两个部分之间提供高带宽、低延迟的连接[31]。该开发板还包含512 MB的DDR3内存和一些其他外设,使用户能够体验嵌入式设计的各个方面。
\begin{figure}
	\centering
	\caption{\song\wuhao 基于FPGA的人脸识别算法加速器}

\end{figure}
查询人脸的行(320×243像素)存储在DDR存储器中,RAM用于存储数据库人脸每行的STD。这些值仅离线计算一次。对于任何新的数据库人脸数据,必须计算每行的STD并存储在RAM中。在图4中,如果我们的数据库存有人脸数据,则RAM的每一行都存有数据库人脸每一行的STD。

	DMA通过高性能端口(HP)将数据从DDR内存传输到其他系统部件,而不需要CPU参与。使用DMA将提高数据吞吐量,并将处理器从涉及内存传输的任务中脱离了出来。处理器需要在开始时对DMA控制寄存器进行编程来启动传输。处理单元(PU)由两个IP块、直方图和标准差(STD)组成。它通过DMA端口从DDR接收查询人脸行数据,并计算该行的直方图,然后计算该行的标准差。功能单元(FU)从PU中获取查询人脸的STD,从块RAM中获取数据库人脸的STD,然后根据(6)计算距离指数。它还将所有行的距离指数相加,找出查询人脸和数据库人脸之间的最后一行,然后将结果存储在寄存器中。例如,Reg1将为每个数据库人脸和第1、5、9等行计算std的和。Reg2将为第2、6、10等行计算结果。比较器存储所有距离指数,并找到最小值,该值表示与查询人脸匹配的数据库人脸。算法2给出了在FPGA-Zed板上实现硬件加速查询人脸与数据库人脸匹配过程的伪码。
	
FPGA开发板中,利用高性能端口(HP)访问外部存储(DDR3)。这些端口被设计为4端口×64比特×150 MHz=4.8GbETE/SEC的最大带宽。为了充分利用带宽,我们使用处理系统的32位数据端口(HP0、HP1、HP2和HP3)从DDR接收数据,并将其视为独立线程。每个线程由计算直方图,标差的处理单元(PU)和计算距离分数的函数单元(FU)组成。因此,我们的加速器可以并行处理四行。

每个线程的执行都经过不同的阶段(见图5):首先,第1行的直方图是通过第一个阶段计算的。其他两个阶段是空的。计算直方图后,将第1行移到第二阶段并计算标准偏差。同时,另一行进入计算直方图阶段,并且很快。一旦DMA交付第3行,三个阶段将同时工作。这将确保高级别的数据并行性并提高执行性能。
\begin{figure}
	\centering
	\caption{\song\wuhao 在硬件加速器中执行该算法的各个阶段}

\end{figure}







